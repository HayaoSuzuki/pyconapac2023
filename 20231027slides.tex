\documentclass[aspectratio=169,dvipdfmx,12pt,notheorems]{beamer}
%%%% 和文用 %%%%%
\usepackage{bxdpx-beamer}
\usepackage{pxjahyper}
\usepackage{minijs}%和文用
\renewcommand{\kanjifamilydefault}{\gtdefault}%和文用

%%%% スライドの見た目 %%%%%
\usetheme{Madrid}
\usefonttheme{professionalfonts}
\setbeamertemplate{frametitle}[default][center]
\setbeamertemplate{navigation symbols}{}
\setbeamercovered{transparent}%好みに応じてどうぞ)
\setbeamertemplate{blocks}[rounded]
\useinnertheme{circles}
\setbeamertemplate{footline}[page number]
\setbeamerfont{footline}{size=\normalsize,series=\bfseries}
\setbeamercolor{footline}{fg=black,bg=black}
%%%%

%%%% 定義環境 %%%%%
\usepackage{amsmath,amssymb}
\usepackage{amsthm}
\theoremstyle{definition}
\newtheorem{theorem}{定理}
\newtheorem{definition}{定義}
\newtheorem{proposition}{命題}
\newtheorem{lemma}{補題}
\newtheorem{corollary}{系}
\newtheorem{conjecture}{予想}
\newtheorem*{remark}{Remark}
\renewcommand{\proofname}{}
%%%%%%%%%

%%%%% フォント基本設定 %%%%%
\usepackage[T1]{fontenc}%8bit フォント
\usepackage{textcomp}%欧文フォントの追加
\usepackage[utf8]{inputenc}%文字コードをUTF-8
\usepackage[deluxe]{otf}%otfパッケージ
\usepackage{lxfonts}%数式・英文ローマン体を Lxfont にする
\usepackage{bm}%数式太字
%%%%%%%%%%

%%%%% PythonTeX %%%%%
\usepackage[makestderr]{pythontex}
\restartpythontexsession{\thesection}
 
\title{Let's implement useless Python objects}
\subtitle{役に立たないPythonオブジェクトを作ろう}
\author[Hayao]{Hayao Suzuki}
\institute[PyCon APAC 2023]{PyCon APAC 2023}
\titlegraphic{\includegraphics[scale=0.2]{pyconlogo.png}}
\date{October 27, 2023}

\begin{document}

\begin{frame}[plain]\frametitle{}
\titlepage %表紙
\end{frame}

\begin{frame}\frametitle{Share it}

\begin{block}{GitHub}
\begin{itemize}
\item \url{https://github.com/HayaoSuzuki/pyconapac2023}
\end{itemize}
\end{block}

\begin{block}{Hashtag}
\begin{itemize}
\item \#pyconapac \#pyconapac2023 \#pyconjp
\end{itemize}
\end{block}

\end{frame}

\section{Introduction}

\begin{frame}\frametitle{Who am I ?}

\begin{block}{Who am I ?(お前誰よ)}
\begin{description}
\item[Name] Hayao Suzuki(鈴木 駿)
\item[X(Twitter)] \href{https://twitter.com/CardinalXaro}{@CardinalXaro}
\item[Work] Software Developer @ BeProud Inc.
\end{description}
\end{block}

\begin{center}
\begin{itemize}
\item BeProud Inc. \includegraphics[width=3cm]{bplogo.png}
\begin{itemize}
\item connpass \includegraphics[width=2cm]{connpass_logo_1.png}
\item PyQ \includegraphics[width=1cm]{pyq_logo_color.png}
\item Tracery \includegraphics[width=3cm]{tracery.png}
\end{itemize}
\end{itemize}
\end{center}

\end{frame}

\begin{frame}\frametitle{Who am I ?}

\begin{block}{Translated Books}
\begin{itemize}
\item \structure{TBA}(O'Reilly Japan) \structure{New!}
\end{itemize}
\end{block}

\begin{block}{Supervised Translated Books}
\begin{itemize}
\item \structure{Introducing Python 2nd ed.}(O'Reilly Japan)
\item \structure{Robust Python}(O'Reilly Japan)
\end{itemize}
\end{block}

\end{frame}

\begin{frame}\frametitle{Who am I ?}

\begin{block}{Selected My Talks}
\begin{itemize}
\item \structure{Symbolic Mathematics using SymPy}(PyCon JP 2018)
\item \structure{How to Use In-Memory Streams}(PyCon JP 2020)
\item \structure{Unknown Evolution of the Built-in Function pow}(PyCon JP 2021)
\end{itemize}
\end{block}
Listed at \url{https://xaro.hatenablog.jp/} .
\end{frame}

\section{Useless!}

\begin{frame}\frametitle{Today's Theme}

\begin{center}
\Huge{Let's implement \structure{useless} Python objects}
\end{center}

\end{frame}

\begin{frame}\frametitle{What is it mean useless?}

\begin{block}{From LDOCE}
\begin{enumerate}
\item not useful or effective in any way
\item (informal) unable or unwilling to do anything properly
\end{enumerate}
\end{block}

\end{frame}

\begin{frame}\frametitle{Is the useless object really useless?}

\begin{block}{From Zhuangzi Ren-jian shi(荘子 人間世篇)}
人皆知有用之用 而莫知無用之用也
\end{block}
Everyone knows the usefulness of the useful, but no one knows the usefulness of the useless.

\end{frame}

\begin{frame}\frametitle{Today's Theme}

\begin{block}{Let's implement useless Python objects}
The useless objects are useless, but how to make a useless object is very useful.
\end{block}

\end{frame}

\section{Basic useless Python objects}

\begin{frame}[fragile]\frametitle{What is a useless Python object?}

\begin{exampleblock}{Example: \texttt{LiarContainer}}
\begin{pygments}{python}
>>> c = LiarContainer(["spam", "egg", "bacon"])
>>> "spam" in c
False
>>> "tomato" in c
True
\end{pygments}
\end{exampleblock}

\end{frame}
\begin{frame}[fragile]\frametitle{What is a useless Python object?}

\begin{exampleblock}{Example: \texttt{FibonacciSized}}
\begin{pygments}{python}
>>> s = FibonacciSized(range(50))
>>> len(s)
12586269025
\end{pygments}
\end{exampleblock}

\end{frame}

\begin{frame}[fragile]\frametitle{What is a useless Python object?}

\begin{exampleblock}{Example: \texttt{ShuffledIterable}}
\begin{pygments}{python}
>>> it = ShuffledIterable([1, 2, 3, 4, 5])
>>> for _ in range(3):
...     for v in it:
...         print(v, end=" ")
...     print()
...
5 3 4 2 1 
4 1 2 3 5 
2 5 3 1 4
\end{pygments}
\end{exampleblock}

\end{frame}

\begin{frame}\frametitle{What is a useless Python object?}

\begin{block}{Definition of a useless Python object in this talk}
A useless Python object behave \structure{Pythonic}, but does not work as expected.
\end{block}

\end{frame}

\begin{frame}[fragile]\frametitle{Data Structures and Operations}

\begin{exampleblock}{Basic Data Structures of Python}
\begin{description}
\item[List] \pyv{[1, 2, 3, 4, 5]}
\item[Tuple] \pyv{("pen", "pineapple", "apple", "pen")}
\item[Dictonary] \pyv{{"Answer": 42}}
\item[Set] \pyv{{41, 43, 47, 53, 57, 59}}
\end{description}
\end{exampleblock}

\begin{exampleblock}{Common Operations of Data Structure}
\begin{description}
\item[\texttt{len()}] Length of object
\item[\texttt{in}] Membership test
\item[\texttt{for}] Iteration
\end{description}
\end{exampleblock}

\end{frame}

\begin{frame}[fragile]\frametitle{\texttt{Useless} Abstract Base Class}

\begin{exampleblock}{Example: \texttt{Useless} ABC}
\begin{pygments}{python}
class Useless(abc.ABC):
    def __init__(self, data: Optional[Iterable] = None):
        if data is not None:
            self._data = [v for v in data]
        else:
            self._data = []
\end{pygments}
\end{exampleblock}
\texttt{Useless} abstract base is useful, contrary to its name.
\end{frame}

\begin{frame}[fragile]\frametitle{\texttt{in} and \texttt{Container}}

\begin{block}{\texttt{object.\_\_contains\_\_()}}
Called to implement membership test operators.
\end{block}

\begin{exampleblock}{Example: \texttt{LiarContainer}}
\begin{pygments}{python}
class LiarContainer(Useless, Container):
    def __contains__(self, item) -> bool:
        return item not in self._data
\end{pygments}
\end{exampleblock}

\end{frame}

\begin{frame}[fragile]\frametitle{\texttt{len()} and \texttt{Sized}}

\begin{block}{\texttt{object.\_\_len\_\_()}}
Called to implement the built-in function \texttt{len()}.
\end{block}

\begin{exampleblock}{Example: \texttt{FibonacciSized}}
\begin{pygments}{python}
class FibonacciSized(Useless, collections.abc.Sized):
    PHI: Final[float] = (1 + math.sqrt(5)) / 2
    def __len__(self) -> int:
        return math.floor(
            (1 / math.sqrt(5)) * pow(self.PHI, len(self._data))
            + (1 / 2)
        )
\end{pygments}
\end{exampleblock}

\end{frame}

\begin{frame}[fragile]\frametitle{\texttt{for} and \texttt{Iterable}}

\begin{block}{\texttt{object.\_\_iter\_\_()}}
Called when an iterator is required for a container. 
\end{block}

\begin{exampleblock}{Example: \texttt{ShuffledIterable}}
\begin{pygments}{python}
class ShuffledIterable(Useless, Iterable):
    def __iter__(self) -> Iterator:
        return iter(random.sample(self._data, k=len(self._data)))
\end{pygments}
\end{exampleblock}

\end{frame}

\begin{frame}\frametitle{Object Protocols}

\begin{block}{How to implement Pythonic Python objects}
We need to understand \structure{object protocols}.
\end{block}
Ref: \url{https://docs.python.org/3/reference/datamodel.html}

\end{frame}

\begin{frame}\frametitle{\texttt{collections.abc}}

\begin{block}{\texttt{collections.abc}}
This module provides \structure{abstract base classes} that can be used to test whether a class provides \structure{a particular interface}.
\end{block}
From \url{https://docs.python.org/3/library/collections.abc.html}

\end{frame}

%\begin{frame}\frametitle{\texttt{collections.abc}}
%
%\begin{exampleblock}{\texttt{collections.abc} and Interface}
%\begin{table}[h]
%\centering
%\begin{tabular}{lcr}
%\hline
%ABC  & Interface \\
%\hline
%\texttt{Sized}  & \texttt{len()} \\
%\texttt{Container}  & \texttt{in} \\
%\texttt{Iterable}  &  \texttt{\_\_iter\_\_()} \\
%\texttt{Collection}  & \texttt{Sized}, \texttt{Container}, \texttt{Iterable}  \\
%\hline
%\end{tabular}
%\end{table}
%\end{exampleblock}
%
%\end{frame}
%
%\begin{frame}\frametitle{\texttt{collections.abc}}
%
%\begin{exampleblock}{\texttt{collections.abc} and Built-in Objects}
%\begin{table}[h]
%\centering
%\begin{tabular}{lcr}
%\hline
%ABC  & built-in objects \\
%\hline
%\texttt{Sequence}  & \texttt{tuple} \\
%\texttt{MutableSequence}  & \texttt{list} \\
%\texttt{MutableSet}  &  \texttt{set} \\
%\texttt{MutableMapping}  & \texttt{dict}  \\
%\hline
%\end{tabular}
%\end{table}
%\end{exampleblock}

%\end{frame}

\end{document}