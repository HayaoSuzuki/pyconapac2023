\documentclass[aspectratio=169,dvipdfmx,14pt,notheorems]{beamer}
%%%% 和文用 %%%%%
\usepackage{bxdpx-beamer}
\usepackage{pxjahyper}
\usepackage{minijs}%和文用
\renewcommand{\kanjifamilydefault}{\gtdefault}%和文用

%%%% スライドの見た目 %%%%%
\usetheme{Madrid}
\usefonttheme{professionalfonts}
\setbeamertemplate{frametitle}[default][center]
\setbeamertemplate{navigation symbols}{}
\setbeamercovered{transparent}%好みに応じてどうぞ)
\setbeamertemplate{blocks}[rounded]
\useinnertheme{circles}
\setbeamertemplate{footline}[page number]
\setbeamerfont{footline}{size=\normalsize,series=\bfseries}
\setbeamercolor{footline}{fg=black,bg=black}
%%%%

%%%% 定義環境 %%%%%
\usepackage{amsmath,amssymb}
\usepackage{amsthm}
\theoremstyle{definition}
\newtheorem{theorem}{定理}
\newtheorem{definition}{定義}
\newtheorem{proposition}{命題}
\newtheorem{lemma}{補題}
\newtheorem{corollary}{系}
\newtheorem{conjecture}{予想}
\newtheorem*{remark}{Remark}
\renewcommand{\proofname}{}
%%%%%%%%%

%%%%% フォント基本設定 %%%%%
\usepackage[T1]{fontenc}%8bit フォント
\usepackage{textcomp}%欧文フォントの追加
\usepackage[utf8]{inputenc}%文字コードをUTF-8
\usepackage[deluxe]{otf}%otfパッケージ
\usepackage{lxfonts}%数式・英文ローマン体を Lxfont にする
\usepackage{bm}%数式太字
%%%%%%%%%%

%%%%% PythonTeX %%%%%
\usepackage[makestderr]{pythontex}
\restartpythontexsession{\thesection}
 
\title{Let's implement useless Python objects}
\subtitle{役に立たないPythonオブジェクトを作ろう}
\author[Hayao]{Hayao Suzuki}
\institute[PyCon APAC 2023]{PyCon APAC 2023}
\date{October 27, 2023}

\begin{document}

\begin{frame}[plain]\frametitle{}
\titlepage %表紙
\end{frame}

\begin{frame}\frametitle{And now...}

\begin{block}{GitHub}
\begin{itemize}
\item \url{https://github.com/HayaoSuzuki/pyconapac2023}
\end{itemize}
\end{block}

\begin{block}{Hashtag}
\begin{itemize}
\item \#pyconapac \#pyconapac2023 \#pyconjp
\end{itemize}
\end{block}

\end{frame}

\section{Introduction}

\begin{frame}\frametitle{Who am I ?}

\begin{block}{Who am I ?(お前誰よ)}
\begin{description}
\item[Name] Hayao Suzuki(鈴木 駿)
\item[X(Twitter)] \href{https://twitter.com/CardinalXaro}{@CardinalXaro}
\item[Work] Software Developer @ BeProud Inc.
\end{description}
\end{block}

\begin{center}
\begin{itemize}
\item BeProud Inc. \includegraphics[width=3cm]{bplogo.png}
\begin{itemize}
\item connpass \includegraphics[width=2cm]{connpass_logo_1.png}
\item PyQ \includegraphics[width=1cm]{pyq_logo_color.png}
\item Tracery \includegraphics[width=3cm]{tracery.png}
\end{itemize}
\end{itemize}
\end{center}

\end{frame}

\begin{frame}\frametitle{Who am I ?}

\begin{block}{Translated Books}
\begin{itemize}
\item \structure{TBA}(O'Reilly Japan) \structure{New!}
\end{itemize}
\end{block}

\begin{block}{Supervised Translated Books}
\begin{itemize}
\item \structure{Introducing Python 2nd ed.}(O'Reilly Japan)
\item \structure{Robust Python}(O'Reilly Japan)
\end{itemize}
\end{block}

\end{frame}

\begin{frame}\frametitle{Who am I ?}

\begin{block}{Selected Presentations at PyCon JP}
\begin{itemize}
\item \structure{Symbolic Mathematics using SymPy}(PyCon JP 2018)
\item \structure{How to Use In-Memory Streams}(PyCon JP 2020)
\item \structure{Unknown Evolution of the Built-in Function pow}(PyCon JP 2021)
\end{itemize}
\end{block}
Listed at \url{https://xaro.hatenablog.jp/} .
\end{frame}

\section{Useless!}

\begin{frame}\frametitle{Today's Theme}

\begin{center}
\Huge{Let's implement \structure{useless} Python objects}
\end{center}

\end{frame}

\begin{frame}\frametitle{What is it mean useless?}

\begin{block}{From LDOCE}
\begin{enumerate}
\item not useful or effective in any way
\item (informal) unable or unwilling to do anything properly
\end{enumerate}
\end{block}

\end{frame}

\begin{frame}\frametitle{Is the useless object really useless?}

\begin{block}{From Zhuangzi Ren-jian shi(荘子 人間世篇)}
人皆知有用之用 而莫知無用之用也
\end{block}
Everyone knows the usefullness of the usefull, but no one knows the usefullness of useless.

\end{frame}


\end{document}